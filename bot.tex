% This is LLNCS.DEM the demonstration file of
% the LaTeX macro package from Springer-Verlag
% for Lecture Notes in Computer Science,
% version 2.4 for LaTeX2e as of 16. April 2010
\documentclass{llncs}

\usepackage{makeidx}  

\begin{document}

\title{Exploratory Population Analysis with Unbalanced Optimal Transport}
\titlerunning{Exploratory Population Analysis with Unbalanced Optimal Transport}  

\author{Samuel Gerber, Marc Niethammer, Martin Styner, Stephen Aylward}
\authorrunning{Gerber et al.} 
\institute{Kitware Inc, Carborro NC 27510, USA,\\
\email{samuel.gerber@kitware.com}
\and
University of North Carolina, Chapel Hill NC 27504, USA}


\maketitle              

% What problems are we addressing?
%  - [MRI] disentangle mass from shape changes
%  - [MRI] No segmentations needed
%  - [VESSELS] unstructured point clouds
%  - [VESSELS] background blood perfusion

% What's new?
%  - Hypothesis generation
%  - Different measure to capture changes
%  - No deformable registration
%  - No parameter tuning
%  - [MRI] Unbalanced mass transport
%  - [VESSELS] Decomposition of transport plan with respect to underlying measure


\begin{abstract}
The plethora of data from neuroimaging studies provide a rich opportunity to
discover effects and generate hypotheses through exploratory data analysis. The
pathologies in brain disease often manifest in changes in shape along with
deterioration and alteration of brain matter, i.e. changes in mass. We propose
to use unbalanced optimal transport to disentangle shape from mass changes and
localize those changes. The exploratory analysis approach generates images of
transport cost and mass changes for each subject in the population.  Using
voxelwise correlations with disease indicators on these images highlight
regions of interest in mass or shape changes related to the disease indicator.
We demonstrate the method on the OASIS brain MRI data set, which includes subject
ranging from healthy to mild and moderate dementia. The results corroborate
known pathology changes related to dementia and suggest avenues for further
clinical research. Additionally regression and permutation testing on the
transport cost and mass change images improve on existing methods to predict
disease measures and indicates that the proposed method captures a larger
portion of the pathology induced changes.  
\end{abstract}

\section{Introduction}

\section{Background}

\subsection{Population Analysis}

\subsection{Optimal Transport}

\section{Optimal Transport for Population Analysis}

\section{Application to OASIS Brain MRI}

\section{Conclusion}

- Localized mass preserving
- Clustering

\end{document}
